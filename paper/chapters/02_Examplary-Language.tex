\chapter{Query Language}

The query language is generic by design and by usage of the newly introduced
generic \cite{go_generic_proposal}. This allows for rudimentary and complex
language variations depending on the data type the desired language is created
with. This allows for a complexity ranging from boolean algebra, over
arithmetic operations to complex queries on lists of objects, such as filtering
for values, mapping and mutating elements of the list and iterating over the
entries of the list.

\begin{listing}[H]
    \begin{minted}{bash}
true | false
a & !b
let c=true;
if c then 
    a&b 
else 
    a|b
    \end{minted}
    \caption{Boolean algebra}
    \label{code:language-example-boolean}
\end{listing}

\begin{listing}[H]
    \begin{minted}{go}
var boolParser = funcGen.New[bool]()
    .AddConstant("false", false)
    .AddConstant("true", true)
    .AddSimpleOp("^", true, 
            func(a, b bool) (bool, error) { return a != b, nil })
    .AddSimpleOp("=", true, 
            func(a, b bool) (bool, error) { return a == b, nil })
    .AddSimpleOp("|", true, 
            func(a, b bool) (bool, error) { return a || b, nil })
    .AddSimpleOp("&", true, 
            func(a, b bool) (bool, error) { return a && b, nil })
    .AddUnary("!", func(a bool) (bool, error) { return !a, nil })
    .SetToBool(func(c bool) (bool, bool) { return c, true })
    \end{minted}
    \caption{Generating boolean algebra}
    \label{code:language-example-boolean-gen}
\end{listing}

\section{Feature set}

\section{Evaluation Approach}
