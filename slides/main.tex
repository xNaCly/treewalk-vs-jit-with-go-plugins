\documentclass{beamer}
\usepackage[ngerman]{babel}
\usepackage[utf8]{inputenc}
\usepackage{graphicx}
\usepackage{hyperref}
\usepackage{minted}
\usemintedstyle{perldoc}
\setcounter{tocdepth}{1}

\title{Comparing a Tree-walk Interpreter with JIT compilation and embedding via Go-plugins}
\subtitle{Evaluating the trade-offs of using the Go-plugin API for JIT compilation while comparing the approach with a Tree-walk interpreter}
\date[2024]{2024}
\author{9525469} 

\institute[DHBW]{DHBW}

\date{\today}

\usetheme{metropolis}
\setbeamertemplate{navigation symbols}{}
\setbeamertemplate{footline}[frame number]
\setbeamertemplate{section in toc}[sections numbered]

\begin{document}
    
    \frame{\titlepage}

    \begin{frame}
        \frametitle{Interpreted Programming Language Performance}

        Common Interpreter design approaches

        \begin{itemize}
            \item AST walker (naive)
            \item Byte code compiler and virtual machine
        \end{itemize}

        Just In Time Compilation
        \begin{itemize}
            \item translating selected byte code or ast nodes to machine code
            \item execute machine code instead of interpreting chunk
        \end{itemize}
    \end{frame}

     \begin{frame}
        \frametitle{Performance Improvements}

        With Just in Time Compilation
        \begin{itemize}
            \item math: $14.51$x improvement (from $13.64$s to $0.94$s)
            \item string concat: $4.59$x improvement (from $14.17$s to $3.09$s)
            \item real world: $1.43$x improvement (from $8.04$s to $5.62$s)
        \end{itemize}
    \end{frame}

    \begin{frame}
        \frametitle{Conclusion}

        \begin{itemize}
            \item implemented JIT for subset of language
            \item measured performance improvements
            \item evaluated the usability of the plugin api
        \end{itemize}
    \end{frame}

    \begin{frame}
        \frametitle{Gophers}
        \begin{figure}
            \includegraphics[scale=8]{./gopher.jpg}
            \caption{Quelle: \texttt{https://go.dev/blog/gopher}}
        \end{figure}
    \end{frame}
\end{document}
